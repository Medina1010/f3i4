\section*{Discusi\'on}

Durante la pr\'actica se observ\'o que, debido al amortiguamiento, al tener un 
movimiento inicial, la amplitud de este decae hasta que su movimiento es igual 
al movimiento forzado. Esta primera fase corresponde al estado transitorio, que 
tras un tiempo determinado evoluciona hacia el estado estacionario. Tambi\'en se 
observ\'o que, al incrementar la corriente del freno y con ello el 
amortiguamiento, el sistema alcanza el estado estacionario en un tiempo menor.

De las gr\'aficas en las figuras (\ref{fig:I02})---(\ref{fig:I08}) se observa que 
la amplitud m\'axima de las oscilaciones disminuye al aumentar la corriente 
aplicada al freno electromagn\'etico. Para \(0{,}2\,\text{A}\) la amplitud 
alcanza aproximadamente \(0{,}207\,\text{m}\), mientras que para 
\(0{,}4\,\text{A}\), \(0{,}6\,\text{A}\) y \(0{,}8\,\text{A}\) los valores se 
reducen a \(0{,}070\,\text{m}\), \(0{,}031\,\text{m}\) y \(0{,}019\,\text{m}\) 
respectivamente. Esta tendencia se explica porque al aumentar la corriente del 
freno tambi\'en lo hace el coeficiente de amortiguamiento \(\gamma\), lo que 
intensifica la disipaci\'on de energ\'ia por las corrientes de Foucault en el 
disco conductor. En estas condiciones, una mayor fracci\'on de la energ\'ia 
suministrada por el momento forzante del motor se disipa en forma de calor, 
reduciendo la amplitud estacionaria incluso cuando el sistema es excitado en la 
frecuencia de resonancia. Este comportamiento es coherente con la teor\'ia del 
oscilador forzado amortiguado, que establece una relaci\'on directa entre la 
amplitud m\'axima y el momento forzante, es inversa con el amortiguamiento del 
sistema.

En la tabla \ref{tab:gammas} se comparan los valores te\'oricos y experimentales de la 
frecuencia de resonancia para corrientes de \(0{,}2\,\text{A}\), 
\(0{,}4\,\text{A}\), \(0{,}6\,\text{A}\) y \(0{,}8\,\text{A}\), junto con sus 
coeficientes de amortiguamiento. Se observa que \(\omega_R\) experimental no 
sigue completamente la tendencia te\'orica, pues entre \(0{,}2\,\text{A}\) y 
\(0{,}6\,\text{A}\) aumenta, cuando deber\'ia disminuir al incrementarse la 
amortiguaci\'on; solo en \(0{,}8\,\text{A}\) se presenta el comportamiento 
esperado. Esta diferencia puede deberse a errores en la medici\'on de la 
frecuencia del motor, a variaciones en la corriente del freno, a la no 
linealidad del acoplamiento mec\'anico o a incertidumbres en la lectura de la 
amplitud, factores que afectan la determinaci\'on precisa de \(\omega_R\) y 
explican la desviaci\'on respecto al modelo te\'orico del oscilador amortiguado.

La gr\'afica de la figura~\ref{fig:amplitudfrecuencia} presenta la variaci\'on de la amplitud m\'axima 
de oscilaci\'on en funci\'on de la frecuencia angular de la fuerza impulsora 
para una corriente de amortiguamiento de \(0{,}4\,\text{A}\). Se observa un 
incremento progresivo de la amplitud a medida que la frecuencia del 
forzamiento se aproxima al valor de resonancia, alcanzando un m\'aximo en 
\(\omega_f = \omega_R = 6{,}504\,\text{s}^{-1}\). Este valor experimental es muy 
pr\'oximo al te\'orico \(\omega_{R,teo} = 6{,}464\,\text{s}^{-1}\), lo que indica 
una correspondencia adecuada entre el modelo del oscilador forzado y los datos 
experimentales obtenidos. La coincidencia entre ambos valores confirma que el 
sistema responde con amplitud m\'axima cuando la frecuencia de excitaci\'on 
coincide aproximadamente con la frecuencia natural, compensando las p\'erdidas 
de energ\'ia debidas al amortiguamiento. Adem\'as, la forma del pico de 
resonancia refleja un r\'egimen de disipaci\'on moderado, consistente con la 
expresi\'on te\'orica \(\omega_R = \sqrt{\omega_0^2 - 2\gamma^2}\).

Las discrepancias observadas entre los resultados experimentales y te\'oricos 
pueden deberse a m\'ultiples factores. Entre los errores mec\'anicos se incluyen 
la fricci\'on en el eje de rotaci\'on, la desalineaci\'on entre el disco y el 
resorte, y el acoplamiento imperfecto entre el motor y el sistema oscilante. En 
el aspecto el\'ectrico, las variaciones en la corriente del freno o en el 
voltaje del motor modifican el momento forzante y el coeficiente de 
amortiguamiento. Tambi\'en pueden presentarse errores por calentamiento en la 
bobina del freno y en la determinaci\'on experimental de la frecuencia del 
motor. Finalmente, las lecturas visuales de la amplitud y las limitaciones del 
sensor de posici\'on introducen incertidumbres adicionales que afectan la 
precisi\'on en la determinaci\'on de \(\omega_R\) y de las amplitudes medidas.
