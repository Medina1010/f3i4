\hrule
\begin{abstract}
El presente informe tuvo como objetivo analizar el comportamiento del p\'endulo de 
Pohl forzado y determinar experimentalmente su frecuencia de resonancia para 
diferentes condiciones de amortiguamiento. Este sistema permite estudiar la 
transici\'on entre el estado transitorio y el estacionario, donde la energ\'ia disipada 
por fricci\'on es compensada por una fuerza externa peri\'odica. El montaje 
experimental consisti\'o en un p\'endulo de tors\'ion con freno electromagn\'etico y 
motor de velocidad variable, registrando la amplitud de oscilaci\'on con el sistema 
CASSY. Se observ\'o que al incrementar la corriente del freno la amplitud m\'axima 
disminuye, indicando una mayor disipaci\'on de energ\'ia, mientras que la resonancia 
se presenta cuando la frecuencia impulsora se aproxima a la frecuencia natural del 
sistema. Los valores experimentales de la frecuencia de resonancia fueron 
coherentes con la teor\'ia, aunque se evidenciaron desviaciones atribuidas a errores 
de medici\'on, fluctuaciones el\'ectricas y acoplamientos mec\'anicos no lineales. En 
conjunto, los resultados confirman el modelo del oscilador forzado amortiguado y 
permiten establecer la relaci\'on entre la frecuencia de resonancia, la constante de 
amortiguamiento y la amplitud estacionaria del sistema.
	\loremlines{10}
	\textbf{Palabras clave: oscilador forzado, resonancia, amortiguamiento, p\'endulo de 
	Pohl, frecuencia angular}
\end{abstract}
\hrule


