\section*{Conclusiones}
El estudio del p\'endulo de Pohl forzado permiti\'o observar la transici\'on del 
sistema desde un r\'egimen transitorio hasta un estado estacionario, donde la 
amplitud se mantiene constante bajo la acci\'on conjunta del amortiguamiento y 
del momento forzante. Los resultados experimentales mostraron una buena 
correspondencia con el modelo te\'orico, aunque se presentaron discrepancias en 
la frecuencia de resonancia al aumentar la corriente de frenado, pues el valor 
experimental de \(\omega_R\) no disminuy\'o de manera uniforme como predice la 
teor\'ia. Este efecto se atribuye a variaciones en la corriente del freno, a la 
no linealidad del acoplamiento mec\'anico y a incertidumbres en la medici\'on de 
la frecuencia. Tambi\'en se comprob\'o que una mayor corriente incrementa la 
disipaci\'on de energ\'ia y reduce la amplitud de oscilaci\'on, mientras que el 
momento forzante del motor determina la respuesta m\'axima alcanzada en la 
resonancia. En conjunto, los resultados confirman la validez general del modelo 
del oscilador forzado amortiguado dentro de los l\'imites experimentales.
