\section*{Anexos}
\subsection*{Regresión lineal para las variables $\mathbf{\omega_0, \omega_R}$ y $\mathbf{\gamma}$}

Para determinar las constantes $\omega_0, \omega_R$ y $\gamma$, se reescribieron las
Ec. (\ref{eq:omegasolito}), (\ref{eq:frecuenciaresonancia}) y (\ref{eq:solucionhomogeneo})
en la forma $y = \beta_0 x + \beta_1$, siendo $\beta_0$ la pendiente y $\beta_1$ el intercepto.
De esta manera, las expresiones
\begin{align*}
	\omega^2 &= \omega_0^2 - \gamma^2 \\
	\omega_0^2 &= \omega_R^2 + 2\gamma^2 \\
	\ln(\theta) &= -\gamma t + \ln(A),
\end{align*}
son adecuadas para un análisis de regresión lineal empleando los datos registrados.
