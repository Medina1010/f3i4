En la Tab. (\ref{tab:gammas}) se presentan los valores de la frecuencia de amortiguamiento $\gamma$ y de resonancia $\omega_R$ correspondientes a cada corriente de amortiguamiento y
voltaje impulsor empleado. La frecuencia natural $\omega_0$ establecida para el péndulo fue
de \qty{6,511}{\per\second} y se usó la Ec. (\ref{eq:frecuenciaresonancia}) para determinar el
valor teórico de $\omega_R$. Por otro lado, en el anexo se especifica el proceso empleado
para determinar $\omega_0$, $\gamma$ y $\omega_R$. 

\begin{table}[H]
	\centering
	\captionof{table}{Frecuencia de resonancia y amplitud máxima para las frecuencias de
					amortiguación.}
	\begin{tabular}{c c c c c c}
		\toprule
		 \multirow{2}{*}{$I$, A} & \multirow{2}{*}{$\gamma$, s$^{-1}$} & \multirow{2}{*}{$V$, V} & \multirow{2}{*}{$A_M$, m} & \multicolumn{2}{c}{$\omega_R$, s$^{-1}$} \\
		\cmidrule(l){5-6}
		& & & & Teo. & Exp. \\
		\cmidrule(r){1-2} \cmidrule(l){3-6}
		0,2 & 0,1232 & 7,30 & 0,2072  & 6,509 & 6,423 \\
		0,4 & 0,2220 & 7,13 & 0,07074 & 6,504 & 6,464 \\
		0,6 & 0,3446 & 7,05 & 0,03154 & 6,493 & 6,481 \\
		0,8 & 0,4783 & 6,95 & 0,01896 & 6,476 & 6,338 \\
		\bottomrule
		\label{tab:gammas}
	\end{tabular}
\end{table}