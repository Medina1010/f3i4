                                %%%%%%%%%%%%%%%
                                % INFORMACION %
                                %%%%%%%%%%%%%%%
\title{
	\LARGE\textbf{
	Análisis del Péndulo de Pohl Forzado: Determinación de su
	Frecuencia de Resonancia Para Múltiples Amortiguamientos
	}
}
\author{
	Christian Camilo Idarraga,
	Juan Diego García,
	Julián Medina
}
\date{\today}
                          %%%%%%%%%%%%%%%%%%%%%%%%%%%%
                          % PAQUETES Y CONFIGURACION %
                          %%%%%%%%%%%%%%%%%%%%%%%%%%%%
% unidades
\usepackage{siunitx}
% geometria
\usepackage{geometry}
\geometry{a4paper, margin=1in}
% graficos
\usepackage{pgf, pgfplots, pgfplotstable}
\pgfplotsset{width=7cm, compat=newest}
% multicolumnas
\usepackage{multicol}
\usepackage{float}
% idioma
\usepackage{booktabs}
\usepackage[spanish]{babel}
\usepackage{csquotes}
% texto de relleno xd
\usepackage{lipsum}
% minipage
\usepackage{graphicx}
\usepackage{subfig}
% simbolos
\usepackage{amsmath}
\usepackage{amssymb}
\usepackage{enumitem}
\usepackage{icomma}
\usepackage{physics}
% referencias
\usepackage[backend=bibtex]{biblatex}
%\usepackage{biblatex}
\addbibresource{res/referencias.bib}
                                   %%%%%%%%%%
                                   % MACROS %
                                   %%%%%%%%%%
% fix de comandos en conflicto                         
\AtBeginDocument{\RenewCommandCopy\qty\SI}
% una columna
\def\beginoneccolumn{\end{multicols}}
\def\endonecolumn{\begin{multicols}{2}}
% para tener una buena referencia de que tan largo sera el resumen
\newbox\one
\newbox\two
\long\def\loremlines#1{%
	\setbox\one=\vbox {%
		\lipsum%
	}
	\setbox\two=\vsplit\one to #1\baselineskip
	\unvbox\two}